% !TEX TS-program = xelatex
% !TEX encoding = UTF-8 Unicode
% !Mode:: "TeX:UTF-8"

\documentclass{resume}
\usepackage{zh_CN-Adobefonts_external} % Simplified Chinese Support using external fonts (./fonts/zh_CN-Adobe/)
%\usepackage{zh_CN-Adobefonts_internal} % Simplified Chinese Support using system fonts
\usepackage{linespacing_fix} % disable extra space before next section
\usepackage{cite}

\begin{document}
\pagenumbering{gobble} % suppress displaying page number

\name{郑明}

\basicInfo{
  \email{fireflyEngineer@gmail.com} \textperiodcentered\ 
  \phone{(+86) 15643214252} \textperiodcentered\ 
 }
 
\section{\faGraduationCap\  教育背景}
\datedsubsection{\textbf{吉林农业大学}, 长春}{2010.9 -- 2015.7}
\textit{本科}\ 电子信息科学与技术

\section{\faUsers\ 项目经历}
\datedsubsection{\textbf{菜鸟供应链}}{2020.3 - 12月外包, 2021.01 -- 至今菜鸟}
\role{前端工程师}{主要负责驿站中台相关业务}
\begin{itemize}
  \item 带领5位同学完成驿站中台域业务
  \item 工程化,推进架构升级、分层,提高代码质量、研发规范、研发速度。
  \item 组件化,组件库搭建,动态表单能力
  \item 地图SDK封装
  \item 解决团队中的疑难问题、降低业务代码复杂度,CR,面试,团队能力提升
\end{itemize}

\datedsubsection{\textbf{杭州威佩科技有限公司}}{2019.9 - 2020.3}
\role{前端工程师}{负责游戏比赛以及工具开发}
\begin{itemize}
    \item 皇室战争擂台赛
    \item 荒野乱斗数据工具 (微信小程序)
    \item 皇室战争数据工具
    \item H5截图工具
    \item 主要技术栈 react、SSR、typescript、koa、nodejs、websocket
    \item 由于所处业务边缘切不盈利,遇新冠疫情被裁
\end{itemize}

\datedsubsection{\textbf{杭州兑吧网络科技有限公司}}{2018.8 - 2019.8}{
\role{前端工程师}{主要负责积分商城Saas服务,以及可视化编辑器开发}
\begin{itemize}
    \item 参与canvas引擎开发,可视化编辑器(electron)开发,相关内容包括,设计时、运行时,引擎解释器,ts declare 编译,组件设计, 虚拟文件系统,线上组组件库服务(golang开发)等整套解决方案,方案类似于cocos。
    \item 包活运营类活动及抽象, 以及 canvas 游戏类运营活动及抽象,实现高效可复用、可配置运营活动。
\end{itemize}
}

\datedsubsection{\textbf{杭州遥望网络有限公司}}{2017.11 - 2018.6}{
\role{前端工程师}{B端广告平台、C端知识付费、数据可视化等}
\begin{itemize}
    \item B端广告主平台搭建 (业务:用户来平台购买广告位)
    \item 基于Docker,Sentry 搭建前端错误监控系统
    \item 基Jenkins搭建自动化流程 (jenkins + 阿里云OSS 文件传输 采用 node.js SDK + shell脚本,git, ssh )
    \item C端变现类项目 (包括技术选型、框架搭建等)
    \item 基于 Echarts 开发数据可视化,动态观测技术内部,BUG、故障等数据,以及用户活跃状态,加强对项目质量以及业务的理解。
\end{itemize}
}

\datedsubsection{\textbf{杭州先手科技有限公司}}{2017.2 - 2017.11}{
\role{前端工程师}{电商商城C端页面、以及管理系统}
\begin{itemize}
    \item 微商、电商系统后台管理,C端业务
    \item 因全部都是重复性工作以及团队氛围等因素个人无法改变后离开。
\end{itemize}
}

\datedsubsection{\textbf{帮家里人做生意}}{{2015.7 - 2017.2}{
\begin{itemize}
    \item 毕业后,对工作比较迷茫,所以选择帮家里人做生意,微商电商相关
    \item 商品(美妆类、服装类等)从研发 -> 生产 -> 货运 -> 仓储 -> 销售 -> 物流 -> 客服 -> 赔付等,生命周期里都很熟悉
    \item 其中找一些软件外包,自己也策划、研发了一些活动等
    \item 后因为兴趣缺乏、已经缺少提升空间、能力凭借,转行技术
\end{itemize}
}}

% Reference Test
%\datedsubsection{\textbf{Paper Title\cite{zaharia2012resilient}}}{May. 2015}
%An xxx optimized for xxx\cite{verma2015large}
%\begin{itemize}
%  \item main contribution
%\end{itemize}

\section{\faCogs\ IT 技能}
% increase linespacing [parsep=0.5ex]
\begin{itemize}[parsep=0.5ex]
  \item 编程语言: javascript、 typescript > node.js > go 
  \item 编程范式: FP > OOP > EOP
  \item 前端技术栈:react > vue
  \item 其他:koa、gin、nginx、docker、k8s等稍有涉及
\end{itemize}

\section{\faHeartO\ 自我评价}
\textit{深入业务,关注并深挖技术底层,较强的分析问题,解决问题能力,良好的团队沟通、协作和服务意识。善于思考问题的细节,有不错的创造力。对于软件工程有着不错的架构分层能力。
热爱生活、热爱工作。}

\section{\faInfo\ 其他}
% increase linespacing [parsep=0.5ex]
\begin{itemize}[parsep=0.5ex]
  \item GitHub: https://github.com/NoahStepheno
\end{itemize}

%% Reference
%\newpage
%\bibliographystyle{IEEETran}
%\bibliography{mycite}
\end{document}
